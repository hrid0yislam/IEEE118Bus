\documentclass[11pt]{article}
\usepackage[utf8]{inputenc}
\usepackage[margin=1in]{geometry}
\usepackage{listings}
\usepackage{xcolor}
\usepackage{hyperref}
\usepackage{amsmath}
\usepackage{graphicx}
\usepackage{float}

% Configure code listing style
\lstset{
    language=Python,
    basicstyle=\ttfamily\small,
    numbers=left,
    numberstyle=\tiny,
    frame=single,
    breaklines=true,
    keywordstyle=\color{blue},
    commentstyle=\color{green!60!black},
    stringstyle=\color{red},
    showstringspaces=false,
    tabsize=4
}

\title{Python Implementation for OpenDSS Analysis\\IEEE 118-Bus System}
\author{Power System Analysis Documentation}
\date{\today}

\begin{document}
\maketitle

\section{Introduction}
This document explains the Python implementation for analyzing the IEEE 118-bus system using OpenDSS. The implementation consists of several key components: initialization, data collection, analysis, and visualization.

\section{Core Implementation}

\subsection{OpenDSS Interface}
The primary interface with OpenDSS is established through the OpenDSSDirect.py package:

\begin{lstlisting}[caption=OpenDSS Interface Setup]
import opendssdirect as dss
import pandas as pd
import matplotlib.pyplot as plt
import seaborn as sns
import numpy as np
\end{lstlisting}

\subsection{System Initialization}
The initialization process sets up the OpenDSS environment and loads the circuit components:

\begin{lstlisting}[caption=System Initialization Function]
def initialize_opendss():
    # Clear any existing circuit
    dss.run_command('Clear')
    
    # Set the frequency
    dss.run_command('Set DefaultBaseFrequency=50')
    
    # Create new circuit
    dss.run_command('New Circuit.ieee118bus basekv=138.0 phases=3 bus1=89_clinchrv')
    
    # Load components
    dss.run_command('Redirect generators.dss')
    dss.run_command('Redirect lines.dss')
    dss.run_command('Redirect transformers.dss')
    dss.run_command('Redirect loads.dss')
    dss.run_command('Redirect shunts.dss')
    
    # Set voltage bases
    dss.run_command('Set VoltageBases=[138.0]')
    dss.run_command('Calcv')
    
    # Solution parameters
    dss.run_command('set algorithm=NCIM')
    dss.run_command('set maxiterations=100')
    dss.run_command('set tolerance=0.0001')
    dss.run_command('set loadmodel=1')
\end{lstlisting}

\section{Data Collection and Analysis}

\subsection{Voltage Data Collection}
The implementation includes functions to collect voltage data from all buses:

\begin{lstlisting}[caption=Voltage Data Collection Function]
def get_voltage_data():
    voltages = []
    bus_names = []
    
    dss.Circuit.SetActiveBus('')
    for bus in dss.Circuit.AllBusNames():
        dss.Circuit.SetActiveBus(bus)
        v_mag = dss.Bus.puVmagAngle()[0]
        voltages.append(v_mag)
        bus_names.append(bus)
    
    return pd.DataFrame({
        'Bus': bus_names,
        'Voltage (pu)': voltages
    })
\end{lstlisting}

\subsection{Voltage Profile Visualization}
The visualization component creates comprehensive plots of the voltage profile:

\begin{lstlisting}[caption=Voltage Profile Visualization Function]
def plot_voltage_profile(df):
    plt.style.use('default')
    plt.rcParams['figure.figsize'] = [15, 10]
    plt.rcParams['figure.dpi'] = 300
    
    fig = plt.figure()
    
    # Bar plot of voltage magnitudes
    plt.subplot(2, 1, 1)
    plt.bar(range(len(df)), df['Voltage (pu)'],
           alpha=0.6, color='skyblue')
    plt.axhline(y=1.05, color='r', linestyle='--',
               label='Upper Limit (1.05 pu)')
    plt.axhline(y=0.95, color='r', linestyle='--',
               label='Lower Limit (0.95 pu)')
    plt.title('Voltage Profile of IEEE 118-Bus System')
    plt.xlabel('Bus Number')
    plt.ylabel('Voltage (pu)')
    plt.legend()
    
    # Voltage distribution histogram
    plt.subplot(2, 1, 2)
    sns.histplot(data=df['Voltage (pu)'],
                bins=30, kde=True)
    plt.axvline(x=1.05, color='r', linestyle='--')
    plt.axvline(x=0.95, color='r', linestyle='--')
    plt.title('Voltage Distribution')
    plt.xlabel('Voltage (pu)')
    plt.ylabel('Count')
    
    plt.savefig('latex_report/voltage_profile.png',
                bbox_inches='tight')
    plt.close()
\end{lstlisting}

\section{Statistical Analysis}
The implementation includes comprehensive statistical analysis:

\begin{lstlisting}[caption=Statistical Analysis Implementation]
def analyze_voltage_statistics(df):
    voltage_status = pd.cut(
        df['Voltage (pu)'],
        bins=[-np.inf, 0.95, 1.05, np.inf],
        labels=['Low', 'Normal', 'High']
    )
    
    status_counts = voltage_status.value_counts()
    total = len(df)
    percentages = (status_counts / total * 100).round(2)
    
    summary_stats = {
        'Mean': df['Voltage (pu)'].mean(),
        'Maximum': df['Voltage (pu)'].max(),
        'Minimum': df['Voltage (pu)'].min(),
        'Std Dev': df['Voltage (pu)'].std()
    }
    
    return status_counts, percentages, summary_stats
\end{lstlisting}

\section{Results Generation}
The implementation includes functions to generate comprehensive reports:

\begin{lstlisting}[caption=Results Generation]
def generate_voltage_summary(df, counts, percentages, stats):
    with open('latex_report/voltage_summary.txt', 'w') as f:
        f.write("Voltage Profile Summary\n")
        f.write("=====================\n\n")
        
        # Write classification results
        summary_df = pd.DataFrame({
            'Count': counts,
            'Percentage (%)': percentages
        })
        f.write(str(summary_df))
        
        # Write statistics
        f.write("\n\nVoltage Statistics:\n")
        f.write("------------------\n")
        for key, value in stats.items():
            f.write(f"{key}: {value:.4f} pu\n")
\end{lstlisting}

\section{Main Execution Flow}
The main execution flow coordinates all components:

\begin{lstlisting}[caption=Main Execution Function]
def main():
    # Initialize OpenDSS
    initialize_opendss()
    
    # Get voltage data
    voltage_data = get_voltage_data()
    
    # Create visualizations
    plot_voltage_profile(voltage_data)
    
    # Perform statistical analysis
    counts, percentages, stats = analyze_voltage_statistics(voltage_data)
    
    # Generate summary
    generate_voltage_summary(voltage_data, counts, percentages, stats)
    
    print("Analysis completed. Check latex_report directory for outputs.")

if __name__ == "__main__":
    main()
\end{lstlisting}

\section{Key Features}
The implementation provides several key features:

\begin{enumerate}
    \item \textbf{Modularity}: Separate functions for different aspects of analysis
    \item \textbf{Data Management}: Efficient handling of voltage data using pandas
    \item \textbf{Visualization}: Comprehensive plotting using matplotlib and seaborn
    \item \textbf{Statistical Analysis}: Detailed voltage profile statistics
    \item \textbf{Report Generation}: Automated generation of analysis reports
\end{enumerate}

\section{Error Handling}
The implementation includes basic error handling:

\begin{lstlisting}[caption=Error Handling Example]
try:
    initialize_opendss()
except Exception as e:
    print(f"Error initializing OpenDSS: {str(e)}")
    error = dss.Error.Description()
    if error:
        print(f"OpenDSS Error: {error}")
    sys.exit(1)
\end{lstlisting}

\section{Conclusion}
This Python implementation provides a comprehensive framework for analyzing power systems using OpenDSS. It combines efficient data processing, detailed analysis, and clear visualization to provide insights into system behavior.

\end{document} 