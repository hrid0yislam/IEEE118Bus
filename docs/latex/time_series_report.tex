\documentclass[11pt]{article}
\usepackage[utf8]{inputenc}
\usepackage[margin=1in]{geometry}
\usepackage{graphicx}
\usepackage{listings}
\usepackage{xcolor}
\usepackage{amsmath}
\usepackage{booktabs}
\usepackage{hyperref}
\usepackage{float}

% Configure code listing style
\lstset{
    language=Python,
    basicstyle=\ttfamily\small,
    numbers=left,
    numberstyle=\tiny,
    frame=single,
    breaklines=true,
    keywordstyle=\color{blue},
    commentstyle=\color{green!60!black},
    stringstyle=\color{red}
}

\title{Time Series Analysis Report:\\IEEE 118-Bus System}
\author{Power System Analysis}
\date{\today}

\begin{document}
\maketitle

\begin{abstract}
This report presents a comprehensive time series analysis of the IEEE 118-bus power system over a 24-hour period. The analysis focuses on voltage profiles, system losses, and load variations, providing detailed insights into system behavior and recommendations for improvements.
\end{abstract}

\tableofcontents

\section{Load Profile Analysis}

\subsection{Daily Load Pattern}
The system exhibits distinct loading patterns throughout the day:
\begin{itemize}
    \item \textbf{Peak Periods}: 11:00 and 18:00 (100\% loading)
    \item \textbf{Minimum Load}: 04:00 (55\% loading)
    \item \textbf{Load Variations}:
    \begin{itemize}
        \item Morning Ramp: 05:00-11:00 (55\% → 100\%)
        \item Evening Ramp: 16:00-18:00 (93\% → 100\%)
        \item Night Decline: 19:00-23:00 (97\% → 68\%)
    \end{itemize}
\end{itemize}

\subsection{Load Multipliers}
The hourly load multipliers used in the simulation:
\begin{lstlisting}
load_multipliers = [0.65, 0.60, 0.58, 0.56, 0.55, 0.57, 0.62, 0.72, 0.85, 
                   0.95, 0.98, 1.00, 0.99, 0.97, 0.95, 0.93, 0.94, 0.98, 
                   1.00, 0.97, 0.92, 0.85, 0.75, 0.68]
\end{lstlisting}

\section{Voltage Profile Analysis}

\subsection{Critical Bus Voltages}
\begin{table}[H]
\centering
\begin{tabular}{lcc}
\toprule
Bus & Base Voltage (pu) & Status \\
\midrule
89\_CLINCHRV & 0.678 & Highest \\
69\_SPORN & 0.161 & Critical \\
77\_TURNER & 0.227 & Critical \\
92\_SALTVLLE & 0.550 & Concerning \\
\bottomrule
\end{tabular}
\caption{Critical Bus Voltage Summary}
\end{table}

\subsection{Regional Distribution}
The system shows significant regional variations in voltage profiles:
\begin{itemize}
    \item \textbf{Northern Region}: 0.161-0.350 pu (Most critical)
    \item \textbf{Central Region}: 0.450-0.550 pu (Below normal)
    \item \textbf{Southern Region}: 0.550-0.678 pu (Best performing)
\end{itemize}

\section{System Losses}

\subsection{Loss Components}
Total system losses can be broken down into:
\begin{itemize}
    \item \textbf{Line Losses}: 76,353.20 kW (99.996\%)
    \item \textbf{Transformer Losses}: 30.54 kW (0.004\%)
    \item \textbf{Total Base Losses}: 76,383.74 kW
\end{itemize}

\subsection{Loss Variation}
Loss characteristics throughout the day:
\begin{itemize}
    \item \textbf{Peak Hours}: ∼76 MW
    \item \textbf{Minimum Load}: ∼23 MW
    \item \textbf{Relationship}: Quadratic with loading (I²R losses)
\end{itemize}

\section{Key Findings}

\subsection{Voltage Violations}
Analysis reveals significant voltage issues:
\begin{itemize}
    \item All buses operate below nominal voltage
    \item No buses within acceptable range (0.95-1.05 pu)
    \item Most severe violations in northern region
    \item Voltage variations follow load pattern
\end{itemize}

\subsection{Loss Characteristics}
The system exhibits the following loss patterns:
\begin{itemize}
    \item Quadratic relationship with loading confirmed
    \item Line losses dominate system losses
    \item Peak losses coincide with maximum loading
    \item Regional loss concentration in heavily loaded areas
\end{itemize}

\subsection{System Performance Metrics}
\begin{itemize}
    \item \textbf{Voltage Stability}: Concerning (all buses below 0.95 pu)
    \item \textbf{Loss Efficiency}: Poor (30.48\% of total power)
    \item \textbf{Regional Disparities}: Significant north-south variation
\end{itemize}

\section{Recommendations}

\subsection{Immediate Actions (1-2 months)}
\subsubsection{Voltage Support}
\begin{itemize}
    \item Install capacitor banks:
    \begin{itemize}
        \item Bus 69\_SPORN: 150 MVAR
        \item Bus 77\_TURNER: 100 MVAR
    \end{itemize}
    \item Adjust transformer taps
    \item Implement local voltage control
\end{itemize}

\subsubsection{Loss Reduction}
\begin{itemize}
    \item Optimize power flow
    \item Balance loading
    \item Monitor critical lines
\end{itemize}

\subsection{Medium-term Solutions (3-6 months)}
\begin{itemize}
    \item Install SVCs at critical buses
    \item Network reconfiguration
    \item Control system improvements
    \item Real-time voltage monitoring
    \item Loss tracking system
\end{itemize}

\subsection{Long-term Improvements (6+ months)}
\begin{itemize}
    \item Network reinforcement
    \item New transmission lines
    \item Substation upgrades
    \item Smart grid integration
    \item Advanced control systems
\end{itemize}

\section{Results Summary}

\subsection{Voltage Performance}
\begin{itemize}
    \item \textbf{Maximum Voltage}: 0.678 pu (Below normal)
    \item \textbf{Minimum Voltage}: 0.161 pu (Critical)
    \item \textbf{Average Voltage}: 0.419 pu (Poor)
    \item \textbf{Violations}: 100\% of buses
\end{itemize}

\subsection{Loss Analysis}
\begin{itemize}
    \item \textbf{Peak Losses}: 76.38 MW
    \item \textbf{Minimum Losses}: 23.06 MW
    \item \textbf{Average Losses}: 49.72 MW
    \item \textbf{Loss Factor}: 0.3048
\end{itemize}

\subsection{System Metrics}
\begin{itemize}
    \item \textbf{Voltage Stability Margin}: Negative
    \item \textbf{Loss Percentage}: 30.48\%
    \item \textbf{Critical Buses}: 3
    \item \textbf{Affected Regions}: All
\end{itemize}

\section{Implementation Details}

\subsection{Simulation Framework}
The time series simulation implements:
\begin{lstlisting}
def run_time_series():
    # Initialize system
    set_base_parameters()
    
    # For each hour
    for hour in range(24):
        # Set load multiplier
        scale_system_load(load_multipliers[hour])
        
        # Solve power flow
        solve_power_flow()
        
        # Collect metrics
        collect_voltage_data()
        calculate_losses()
        store_results()
\end{lstlisting}

\subsection{Analysis Parameters}
Key simulation parameters:
\begin{itemize}
    \item \textbf{Time Steps}: 24 hours
    \item \textbf{Base Voltage}: 138 kV
    \item \textbf{Monitoring Points}: 118 buses
    \item \textbf{Critical Buses}: 4 monitored locations
\end{itemize}

\section{Future Work}

\subsection{Analysis Extensions}
Potential areas for further investigation:
\begin{itemize}
    \item Dynamic stability assessment
    \item Contingency analysis
    \item Economic impact study
    \item Reliability evaluation
\end{itemize}

\subsection{Tool Enhancements}
Proposed improvements to analysis tools:
\begin{itemize}
    \item Real-time monitoring capabilities
    \item Automated reporting systems
    \item Advanced visualization techniques
    \item Predictive analytics integration
\end{itemize}

\section{Conclusion}
The time series analysis reveals significant voltage and loss variations throughout the day in the IEEE 118-bus system. Critical areas require immediate attention, particularly buses 69\_SPORN and 77\_TURNER. The proposed solutions, implemented in phases, should significantly improve system performance and stability.

\end{document} 