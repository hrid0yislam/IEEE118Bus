% IEEE 118-Bus System Analysis Report
\documentclass[11pt]{article}
\usepackage[utf8]{inputenc}
\usepackage[margin=1in]{geometry}
\usepackage{booktabs}
\usepackage{longtable}
\usepackage{graphicx}
\usepackage{xcolor}
\usepackage{colortbl}
\usepackage{float}
\usepackage{hyperref}

\title{IEEE 118-Bus System Analysis Report}
\author{Power System Analysis Report}
\date{\today}

\begin{document}
\maketitle

\section{Executive Summary}
This report presents a comprehensive analysis of the IEEE 118-bus system, focusing on voltage profiles, bus characteristics, and system performance. The analysis reveals significant voltage deviations across the system, with all buses operating below nominal voltage levels.

\section{System Overview}
The IEEE 118-bus system is a standard test case representing a portion of the American Electric Power System (in the Midwestern US) as of December 1962. The system operates at a nominal voltage of 138 kV.

\section{Voltage Profile Analysis}

\subsection{System-wide Statistics}
\begin{itemize}
    \item Maximum Voltage: 0.3919 pu at bus 89\_clinchrv (54.08 kV)
    \item Minimum Voltage: 0.0518 pu at bus 1\_riversde (7.15 kV)
    \item System Losses: 76,353.20 kW + j272,041.63 kVAR
\end{itemize}

\section{Bus Analysis}
\begin{longtable}{p{3cm}p{2cm}p{2cm}p{2cm}p{4cm}}
\toprule
Bus Name & Base kV & PU Voltage & Actual kV & Status \\
\midrule
\endhead
89\_clinchrv & 138.0 & 0.3919 & 54.08 & Low Voltage \\
90\_holston & 138.0 & 0.3599 & 49.67 & Low Voltage \\
91\_holstont & 138.0 & 0.3433 & 47.38 & Low Voltage \\
88\_fremont & 138.0 & 0.3523 & 48.62 & Low Voltage \\
85\_beaverck & 138.0 & 0.3043 & 41.99 & Low Voltage \\
87\_pinevle & 138.0 & 0.3019 & 41.66 & Low Voltage \\
1\_riversde & 138.0 & 0.0518 & 7.15 & Low Voltage \\
2\_pokagon & 138.0 & 0.0522 & 7.20 & Low Voltage \\
3\_hickryck & 138.0 & 0.0526 & 7.26 & Low Voltage \\
117\_corey & 138.0 & 0.0521 & 7.19 & Low Voltage \\
\bottomrule
\end{longtable}

\section{Regional Analysis}
\subsection{Higher Voltage Region (>0.30 pu)}
The following buses maintain relatively higher voltages compared to the rest of the system:
\begin{itemize}
    \item Bus 89\_clinchrv (0.3919 pu) - Swing Bus
    \item Bus 90\_holston (0.3599 pu)
    \item Bus 91\_holstont (0.3433 pu)
    \item Bus 88\_fremont (0.3523 pu)
    \item Bus 85\_beaverck (0.3043 pu)
    \item Bus 87\_pinevle (0.3019 pu)
\end{itemize}

\subsection{Lower Voltage Region (<0.06 pu)}
The following buses experience severe voltage depression:
\begin{itemize}
    \item Bus 1\_riversde (0.0518 pu)
    \item Bus 2\_pokagon (0.0522 pu)
    \item Bus 3\_hickryck (0.0526 pu)
    \item Bus 117\_corey (0.0521 pu)
\end{itemize}

\section{System Issues and Recommendations}

\subsection{Identified Issues}
\begin{enumerate}
    \item Severe voltage depression across all buses
    \item Significant deviation from nominal voltage (138 kV)
    \item High system losses
    \item Potential reactive power imbalance
\end{enumerate}

\subsection{Recommendations}
\begin{enumerate}
    \item Verify generator voltage setpoints and control settings
    \item Review transformer tap settings
    \item Evaluate reactive power compensation devices
    \item Consider adding voltage support at critical buses
    \item Investigate load modeling parameters
    \item Review power flow distribution
\end{enumerate}

\section{Conclusion}
The IEEE 118-bus system currently exhibits significant voltage stability issues. All buses are operating below the acceptable voltage range (0.95-1.05 pu), indicating a need for immediate corrective actions. The system requires comprehensive voltage support and reactive power management strategies to improve its operational performance.

\end{document}
