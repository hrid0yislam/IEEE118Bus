\documentclass[11pt]{article}
\usepackage[utf8]{inputenc}
\usepackage[margin=1in]{geometry}
\usepackage{graphicx}
\usepackage{booktabs}
\usepackage{float}
\usepackage{siunitx}
\usepackage{amsmath}
\usepackage{listings}
\usepackage{xcolor}
\usepackage{hyperref}

\lstset{
    language=Python,
    basicstyle=\ttfamily\small,
    numbers=left,
    numberstyle=\tiny,
    frame=single,
    breaklines=true,
    keywordstyle=\color{blue},
    commentstyle=\color{green!60!black},
    stringstyle=\color{red},
    showstringspaces=false,
    tabsize=4
}

\title{Baseline Analysis Report: IEEE 118-Bus System\\Pre-DER Integration Study}
\author{Power System Analysis Report}
\date{\today}

\begin{document}
\maketitle

\section{Executive Summary}
This report presents a comprehensive analysis of the IEEE 118-bus system in its current state, prior to the integration of PV smart inverters. The analysis reveals significant voltage violations across the system, with all buses operating below the acceptable voltage range. This baseline study will serve as a reference point for evaluating the effectiveness of future DER integration.

\section{System Configuration}
\subsection{Base Parameters}
\begin{itemize}
    \item Base Voltage: 138 kV
    \item System Frequency: 50 Hz
    \item Number of Buses: 118
    \item Reference Bus: 89\_clinchrv (swing bus)
\end{itemize}

\subsection{Key Components}
\begin{itemize}
    \item Generators: 54 units (most in constant power factor mode)
    \item Loads: 91 loads at 138 kV level
    \item Transformers: 9 two-winding transformers
    \item Shunt Devices: 14 fixed shunt elements
\end{itemize}

\section{Voltage Profile Analysis}
\subsection{Statistical Summary}
Current voltage statistics show severe undervoltage conditions:
\begin{itemize}
    \item Mean Voltage: 0.1916 pu
    \item Maximum Voltage: 0.6778 pu
    \item Minimum Voltage: 0.0843 pu
    \item Standard Deviation: 0.1463 pu
\end{itemize}

\subsection{Voltage Classification}
The voltage profile classification reveals:
\begin{itemize}
    \item Low Voltage Buses (<0.95 pu): 118 (100\%)
    \item Normal Voltage Buses (0.95-1.05 pu): 0 (0\%)
    \item High Voltage Buses (>1.05 pu): 0 (0\%)
\end{itemize}

\section{Critical Bus Analysis}
\subsection{Best Performing Buses}
Top 5 buses with highest voltage levels:
\begin{enumerate}
    \item Bus 89\_clinchrv: 0.3913 pu (54.01 kV)
    \item Bus 90\_holston: 0.3590 pu (49.55 kV)
    \item Bus 88\_fremont: 0.3517 pu (48.53 kV)
    \item Bus 91\_holstont: 0.3424 pu (47.25 kV)
    \item Bus 92\_saltvlle: 0.3174 pu (43.80 kV)
\end{enumerate}

\subsection{Worst Performing Buses}
Bottom 5 buses with lowest voltage levels:
\begin{enumerate}
    \item Bus 1\_riversde: 0.0486 pu (6.71 kV)
    \item Bus 117\_corey: 0.0490 pu (6.76 kV)
    \item Bus 2\_pokagon: 0.0491 pu (6.77 kV)
    \item Bus 3\_hickryck: 0.0493 pu (6.81 kV)
    \item Bus 13\_concord: 0.0496 pu (6.85 kV)
\end{enumerate}

\section{System Losses}
Current system losses are significant:
\begin{itemize}
    \item Active Power Losses: 75,072.73 kW
    \item Reactive Power Losses: 275,444.49 kVAR
\end{itemize}

\section{Key Code Implementation}
\subsection{Voltage Data Collection}
The following code segment is crucial for monitoring voltage profiles:
\begin{lstlisting}
def get_voltage_data():
    voltages = []
    bus_names = []
    
    dss.Circuit.SetActiveBus('')
    for bus in dss.Circuit.AllBusNames():
        dss.Circuit.SetActiveBus(bus)
        v_mag = dss.Bus.puVmagAngle()[0]
        voltages.append(v_mag)
        bus_names.append(bus)
    
    return pd.DataFrame({
        'Bus': bus_names,
        'Voltage (pu)': voltages
    })
\end{lstlisting}
This function will be essential for comparing voltage profiles before and after DER integration.

\subsection{Statistical Analysis Implementation}
The statistical analysis code provides comprehensive metrics:
\begin{lstlisting}
def analyze_voltage_statistics(df):
    voltage_status = pd.cut(
        df['Voltage (pu)'],
        bins=[-np.inf, 0.95, 1.05, np.inf],
        labels=['Low', 'Normal', 'High']
    )
    
    status_counts = voltage_status.value_counts()
    total = len(df)
    percentages = (status_counts / total * 100).round(2)
    
    summary_stats = {
        'Mean': df['Voltage (pu)'].mean(),
        'Maximum': df['Voltage (pu)'].max(),
        'Minimum': df['Voltage (pu)'].min(),
        'Std Dev': df['Voltage (pu)'].std()
    }
    
    return status_counts, percentages, summary_stats
\end{lstlisting}

\section{Current System Issues}
\subsection{Voltage Control}
\begin{enumerate}
    \item Disabled automatic voltage control
    \item Generator control mode set to constant power factor
    \item Fixed tap positions on transformers
    \item Limited reactive power support
\end{enumerate}

\subsection{System Configuration Issues}
\begin{enumerate}
    \item Swing bus (89\_clinchrv) commented out in generators file
    \item Very low participation factor (0.005) for generators
    \item Many generators set to minimal power output (0.001 kW)
    \item Fixed shunt compensation without dynamic adjustment
\end{enumerate}

\section{Recommendations for DER Integration}
Based on the current system analysis, the following aspects should be considered for PV smart inverter integration:

\subsection{Target Areas}
\begin{enumerate}
    \item Focus on buses with severe undervoltage (<0.05 pu)
    \item Consider proximity to existing generators
    \item Evaluate local load concentrations
    \item Assess network topology for optimal placement
\end{enumerate}

\subsection{Monitoring Points}
Key metrics to monitor after DER integration:
\begin{enumerate}
    \item Voltage profile improvement
    \item System losses reduction
    \item Power flow redistribution
    \item Reactive power compensation
    \item Voltage stability margins
\end{enumerate}

\section{Conclusion}
The current IEEE 118-bus system exhibits severe voltage violations with all buses operating below acceptable limits. The mean voltage of 0.1916 pu indicates a critical need for voltage support. This baseline analysis provides crucial reference points for evaluating the effectiveness of future PV smart inverter integration.

\end{document} 